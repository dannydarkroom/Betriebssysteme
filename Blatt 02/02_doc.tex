%%%%%%%%%%%%%%%%%%%%%%%%%%%%%%%%%%%%%%%%%%%%%%%%%%%%%%%%%%%%%%%%%%%%%%%%%%%%%%%%%%%%%%%%%
%             template.tex                                                              %
%                                                                                       %
%            Author: Sergej Lewin 10/2008                                               %
%                                                                                       %    
% !!!Man braucht noch die Datei Ueb.sty (im gleichen Ordner wie die Hauptdatei)!!!      %
%%%%%%%%%%%%%%%%%%%%%%%%%%%%%%%%%%%%%%%%%%%%%%%%%%%%%%%%%%%%%%%%%%%%%%%%%%%%%%%%%%%%%%%%%
\documentclass[a4paper,11pt]{article}             % bestimmt das Aussehen eines Dokuments
\usepackage{Ueb}                                  % vordefinierte Makros

%!!!!anpassen an das Betriebssystem!!!, um Umlaute zu verwenden
\usepackage[utf8]{inputenc}                      %Linux
%\usepackage[latin1]{inputenc}                    %Windows
%\usepackage[applemac]{inputenc}                  %Mac

%individuelle Pakete
\usepackage{alltt}
\usepackage{listings}
\usepackage{enumitem}
\usepackage{color}
\usepackage[svgnames]{xcolor} 


%Namen und Matrikelnummern anpassen
%\zweinamen{Name1}{Matrikelnummer1}{Name2}{Matrikelnummer2} %2er Gruppen
\dreinamen{Steen Sziegaud}{384705}{Tim Altenburg}{420632}{Stephan Thiemicke}{375485} %3er Gruppe

%Briefkastennummer anpassen. z. B. \briefkasten{104}
\briefkasten{}

%Termin der Uebungsgruppe und Raum anpassen z. B. \termin{Mo. 12-14 , SR2}
\termin{Mi. 16-18Uhr, N1}

%Blattnummer anpassen z. B. \blatt{5}
\blatt{2}

\begin{document}
%Hier kommt der Text des Dokuments......
\Aufgabe{3}

Wir haben die Beobachtung gemacht, dass die Adressen sich bei mehrmaligem ausf"uhren im gdb
nicht "andern, wohingegen diese sich jeweils zwischen zwei Ausf"uhrungen in der
prompt "andern. Dies trifft bei uns sowohl auf den Heap als auch auf den Stack
zu. 
Wir erhielten folgende Ausgaben:
\begin{lstlisting}
# mit gdb:
Value at 0x7fffffffd9d0 (stack) is 42
Value at 0x602010 (heap) is 1337

# ohne gdb:
# 1.
Value at 0x7fff6766d300 (stack) is 42
Value at 0x206e010 (heap) is 1337

# 2.
Value at 0x7fff91b51f00 (stack) is 42
Value at 0x928010 (heap) is 1337

# 3.
Value at 0x7fff96800ec0 (stack) is 42
Value at 0xa1f010 (heap) is 1337
\end{lstlisting}

Damit verhalten sich sowohl Stack als auch (obwohl durch die Fragestellung
anders suggeriert) Heap bei Ausf"uhrung mit bzw. ohne gdb anders.

Da wir die Details der Implementierung weder von C noch vom gdb kennen, k"onnen
wir nur Vermutungen anstellen, warum dieses Verhalten eintrifft.

Wir vermuten, dass der Prozess, wenn er mit gdb gestartet wird, als Unterprozess
gestartet wird. Dabei k"onnte gdb so vorgehen, dass es in seinem eigenen
Adressraum einen Teil f"ur den gestarteten Prozess als Adressraum zur Verf"ugung stellt. Der
gestartete Prozess arbeitet also auf einem Teilbereich des Adressraumes vom gdb
Prozess. Damit h"angen die Adressen des gestarteten Prozesses relativ vom
Adressraum des gdb Prozesses ab. Der gdb scheint relativ zu seinem Adressbereich
den Prozess immer im selben Bereich zu starten, daher erh"alt man in der selben
gdb Session (aber nicht zwischen zwei gdb Sessions) die selben Adressen. 

Dies erkl"art aber noch nicht das Verhalten der dynamischen Speicherverwaltung
im Heap. 
Wir vermuten, dass sich die Adressen in einer gdb Session auch im Heap nicht
"andern, weil die gr"o\ss e des zu allozierenden Speichers nicht gro\ss genug
ist. Da wir nur Speicherplatz f"ur einen int ben"otigen, kann dieser vermutlich
im eigenen Speicherbereich freigegeben werden. Dieser Speicherbereich ist immer
wieder bei einem Neustart verf"ugbar und wird daher zur ersten Allozierung
genutzt, was dazu f"uhrt, dass wir die selben Adressen erhalten. 
W"urden wir einen gr"o\ss eren Speicherplatz ben"otigen, so dass bspw. der
eigene Speicherbereich nicht mehr aussreicht, so m"usste eine Anfrage an das
Betriebssystem geschickt werden, die Speicherplatz f"ur den Prozess zur
Verf"ugung stellt. Dies w"urde in der Regel zu unterschiedlichen Adressen
f"uhren. 

Das Verhalten beim Starten ohne gdb, l"asst sich einfach damit erkl"aren, dass
bei jedem Ausf"uhren ein neuer Prozess mit eigenem Adressraum gestartet wird.
Das Betriebssystem ist daf"ur zust"andig diesen Adressraum zu bestimmen und f"ur
diesen Prozess freizugeben. 


Zwei gleichzeitig ge"offnete Terminal-Fenster:

Wir erhalten beim Starten des zweiten Prozesses einen Segmentation
fault. Dies l"asst sich damit erkl"aren, dass die beiden Prozesse nicht auf den
Speicherbereich des jeweils anderen zugreifen d"urfen. Das ist auch das zu erwartende
Verhalten. 


\Aufgabe{4}

\begin{tabular}{lccccccccccccccccc}
    & 30 & 31 & 32 & 33 & 11 & 12 & 13 & 21 & 22 & 14 & 34 & 41 & 42 & 23 & 35 & 36 &
    43 \\
    $P_1$: & N & R & R & A & A & A & B & R & A & N & N & N & N & N & N & N & N \\
    $P_2$: & N & N & N & N & R & R & A & A & B & B & B & R & A & N & N & N & N \\
    $P_3$: & A & A & A & B & B & B & B & R & R & A & B & R & R & A & A & N & N \\
    $P_4$: & N & N & R & R & R & R & R & R & R & R & A & A & B & B & R & A & N \\
\end{tabular}
    

\Aufgabe{5}

\begin{enumerate}[label=(\alph*)]
    \item $A < B$
    \item die beiden Prozessabschnitte werden nicht ausgef"uhrt, da sie jeweils
        darauf warten, dass $s1$ bzw. $s2$ gesetzt wird. Je nach Definition
        k"onnte man $A < B \wedge B < A$ schreiben.
    \item $A < B \vee B < A$
    \item $A \leq B \vee B \leq A$
    \item $(A < C \wedge A < B \wedge (C \leq B \vee B \leq C)) \vee (B < C
        \wedge B < A \wedge (C \leq A \vee A \leq C))$
    \item $A \leq B \vee B \leq A$
\end{enumerate}

\end{document}
